\documentclass[11pt]{article}

%\usepackage{psfig}
\usepackage{epsfig}

\newcommand{\Wskip}[1]{ }
\newcommand{\Wceil}[1]{\lceil #1 \rceil}
\newcommand{\Wfloor}[1]{\lfloor #1 \rfloor}

\newenvironment{routdef}[1]
{
   \begin{list}{BLANK}
   {
      \setlength{\parsep}{0in}
      \setlength{\itemsep}{.01in}
      \setlength{\partopsep}{0in}
      \setlength{\topsep}{0.1in}
      \setlength{\labelsep}{0in}
      \setlength{\labelwidth}{#1in}
      \setlength{\leftmargin}{#1in}
   }
} {\end{list}}
\newcommand{\rditem}[2]{\item[#1\hfill(~]#2 )}



\textwidth=6in
\textheight=8.7in
\hoffset = -.6in
\voffset = -.6in

\begin{document}

\begin{titlepage}
\title{A Collaborative guide to ATLAS Development}
\vspace{.4in}
\author
{
 R. Clint Whaley \thanks { {\tt rwhaley@users.sourceforge.net} }
 Peter Soendergaard \thanks { {\tt soender@users.sourceforge.net} }
}
\end{titlepage}
\maketitle
\begin{abstract}
This paper exists to get you started if you want to do some ATLAS development.
The hope is that as new developers work on ATLAS, they will add to this note,
so that this information grows with time.
\end{abstract}

\newpage
\tableofcontents
%\listoftables
%\listoffigures

\newpage
\section{Introduction}
This note exists to get you started if you want to write {\bf new}
code for ATLAS, or if you want to modify ATLAS source.  It is {\em not}
for kernel contribution, which is what most people do when contributing to
ATLAS.  Kernel contribution is much simpler, and is explained in
{\tt ATLAS/doc/atlas\_contrib.ps} or in html format at:
\begin{verbatim}
   http://math-atlas.sourceforge.net/atlas_contrib
\end{verbatim}

So what is the difference between development and kernel contribution?  
In kernel
contribution, you write a kernel to be used by ATLAS, using the provided
ATLAS testers and timers to verify it, and when you are satisfied with its
performance and reliability, you submit your kernel to the ATLAS team,
and they accept it or not, and you are done.

Doing actual core development is quite a bit more complex.  Probably the
biggest change is that you will need to write your own tester and timer
for your new code.  No code will be accepted into the ATLAS code base without
a tester which can be used to verify it.  Since writing a decent tester is
usually at least as hard as writing the code it tests, and is always a
whole lot less enjoyable, the author must bear the pain of producing it
along with the pride of producing the code.  As a developer, you will be
responsible for testing such new code on several platforms as well.

If you are instead hoping to modify some of the existing code base, remember
that for non-kernel operations, portability and robustness {\em must} be
the primary goal.  There are many sections of ATLAS that we know to be
second rate on a certain platform, but we also know that it works on the
twenty or so architectures that ATLAS is routinely compiled for, so we leave
it that way.  This means that when a modification is made to a previously
existing routine, the modifying author must have good evidence that
the new code is as portable as the old.  In short, the barrier to replacing
tested code is high.

It is possible that users want access to the CVS repository even though they
do not plan on doing development, mainly I'd guess so they can get access
to the newest stuff without waiting for developer releases.  Also,
kernel contributors who make subsequent changes to their routines can speed up
their adoption by submitting them in a format ready for CVS check-in.

\section{Adding to this note}
This note is included in the {\tt AtlasBase/TexDoc} of the CVS repository,
and anyone can submit a patch against it giving additional information.
As the founder of ATLAS, I have written a seed of a document explaining
how to get access to the code.  It is my hope that other developers will
add important information that they discover as they go, so that this doc
will grow over time, getting much information that I probably take so much
for granted that I would never think to document.

Therefore, new sections are welcome, and probably a FAQ appendix would
be a good idea.  As people contribute, their names will be added to the
author list.

\section{Getting the ATLAS code through CVS}

\subsection{Background on ATLAS code base}

ATLAS was not originally developed in CVS.  ATLAS was developed using a
programming tool called {\em extract}, which means ATLAS is actually
maintained in something called basefiles.  If you think
of regular development being access by value, and CVS using a level of
indirection, CVS on basefiles gives you two such levels of indirection.  So,
if you want to be able to use CVS check-ins, you will need to learn at least
the basics of extract.  Details on extract can be found at:
\begin{verbatim}
   http://www.cs.utsa.edu/~whaley/extract/Extract.html
\end{verbatim}

Note that if you just want read-only access, you will
need to install extract so that you can get at the files, but you will
not need to learn anything about it.

\subsection{Getting the ATLAS CVS tree and the working ATLAS directory}
Here are the steps to get a local ATLAS CVS tree:
\begin{enumerate}
   \item {\bf Install extract}:
Download and install extract as described on the extract homepage.

\item {\bf Checkout the ATLAS CVS tree}:
From a CVS-capable machine connected to the Internet, go to the directory
where you want the ATLAS {\em basefiles} to be, and issue:
\begin{verbatim}
cvs -z3 -d:pserver:anonymous@math-atlas.cvs.sourceforge.net:/cvsroot/math-atlas \
   checkout AtlasBase
\end{verbatim}

\item {\bf Update {\tt topd}}:
Edit the created {\tt AtlasBase/make.base}, and change the definition of
the {\tt topd} macro to match where you have put the AtlasBase directory.
For example, if I issued the CVS command in {\tt /home/rwhaley/work},
I change line 46 from something like:
\begin{verbatim}
   @define topd @/home/rwhaley/Base/AtlasBase@
\end{verbatim}
to:
\begin{verbatim}
   @define topd @/home/rwhaley/work/AtlasBase@
\end{verbatim}

\item {\bf Create ATLAS working directory and extract files}:
Go to the directory where you want the working copy of the {\tt ATLAS/}
directory to reside. In there, issue these two commands:
\begin{verbatim}
   extract -b <your topd>/make.base -o Makefile rout=Make.atldir
   make
\end{verbatim}
\end{enumerate}

\subsection{Basefile/extracted file interaction}
You'll have to scope the extract page for any kind of real feel for how
this works, but some atlas-specific details are in order here.
In each subdirectory of the {\tt ATLAS/} tree, you will find
a file {\tt Make.ext}.  If you type {\tt make -f Make.ext}, this
makefile will extract all new files in this subdirectory for which the
basefile is newer.

So, what usually happens is you a messing with something, and you do it in
the {\tt ATLAS/} directory.  When you are confident in your change,
you put it into the appropriate basefile in the AtlasBase/ directory
(note that examining {\tt Make.ext} will show you what basefile a given
extracted file comes from), and you then re-extract over your working
copy with the above command.

\section{ATLAS coding style}

ATLAS attempts to use a unified coding style.  Older codes may not adhere
to it strictly (especially in function prototyping and single-line ifs), 
but all new codes should do so.

\begin{itemize}
\item Strict ANSI/ISO 9899-1990 C conformance is \underline{required}
      \begin{itemize}
      \item Must use \verb+/* */+, not \verb+//+
      \end{itemize}
\item Lines are at most 80 characters in length
\item No tabs should be used in the file, only spaces
      \vspace*{-0.1in}
      \begin{itemize}
      \item [$\rightarrow$] in vim can use {\tt :set expandtab}
      \end{itemize}
\item Indentation is always 3 characters, and the braces should line
      up, as in:
      \vspace*{-0.1in}
      \begin{verbatim}
      for (i=0; i < N; i++)
      {
         for (j=0; j < N; j++)
            statement;
      }
      \end{verbatim}
      All braces are lined up this way (eg., ifs, functions \& loops).
\item Multiline function calls have arguments indented to the opening paren:
      \vspace*{-0.1in}
      \begin{verbatim}
               error = function_call(arg1, arg2, arg3,
                                     arg4, ar5);
      \end{verbatim}

\item If bodies are always on a separate line from if, so you want:
      \vspace*{-0.1in}
      \begin{verbatim}
          if (cond)
             statement;
      \end{verbatim}
      \vspace*{-0.3in}
      rather than:
      \vspace*{-0.1in}
      \begin{verbatim}
          if (cond) statement;
      \end{verbatim}
\item Comments are either to the left of a line if there is room, or on
      lines preceding the commented code if not, using a style like:
      \vspace*{-0.1in}
      \begin{verbatim}
/*
 *       This comment describes what is going on the below loop body
 *       and if conditional
 */
         if (cond)
         {
         }
/*
 *       Comment describing else body
 */
         else
         {
         }
      \end{verbatim}
      \vspace*{-0.2in}
       Note that the comment text is lined up with the code it comments.
\item Occasionally you need a superblock of comments, that describes a whole
      region of code (eg, multiple if/else blocks).  This is done as:
      \vspace*{-0.1in}
\begin{verbatim}
/*
 * ======================================================================
 * Here is a comment describing the total operation of multiple blocks of
 * code which are all individually commented in the above manner
 * ======================================================================
 */

   for ()
   {
      if (cond)
      ....
      else if
      ...
   }
\end{verbatim}
\item Functions prologues are commented like this:
\begin{verbatim}
int my_func
(
   int iparm,           /* comment describing this parameter */
   void *vp,            /* comment describing this paraemter */
   ...
)
/*
 * This mandatory block of comments placed between the func declaration and
 * the opening brace describes what the function does.  If the function
 * returns a value, this block of comments end with line(s) describing the
 * the return value, of the form:
 * RETURNS: 0 on success, non-zero error message otherwise
 *          second line describing return value
 */
{
}
\end{verbatim}

\end{itemize}

\section{Getting CVS write access/Becoming an ATLAS SourceForge developer}
In a colossal case of lunacy, the developers of SourceForge decided to allow
anyone associated with a SourceForge project full write access to all CVS 
files.  The permissions that can be granted or taken away are the right to
administer lists, changes html docs, do file releases, etc.  To me, this is
a little like having a bank that encrypts your name and password with the
best available means, but leaves your money out in open bins on the front
lawn.

Normally, I would say the more the merrier in terms of adding people as
SourceForge developers.  I'd be happy to see hundreds of people associated
with the project.  On the other hand, I'd be a little scared with having
hundreds of people I've never met have full write access to a project as
detailed and delicate as ATLAS.  CVS has marvelous rollback abilities, but
I'm afraid as of right now anyway, I don't have marvelous CVS abilities,
and so I tend to err on the side of timidity.

What all this whining is coming down to is, you have to first show me the 
money before I'll add you as a developer :).  When you first begin hacking
ATLAS, the method to use is to submit patches or code to the list,
and if your submission is incorporated into ATLAS, we can then get you
added as a SourceForge developer.

\subsection{ATLAS developer guidelines}
We're all still learning, but here are the rough guidelines that you'll
need to be OK with to be added as an ATLAS SourceForge developer:
\begin{enumerate}

\item {\bf You only directly commit to basefiles you originate.}  Assuming
you have written something for ATLAS before, you should have your own
source to maintain.  For instance, if you add some new kernels, we'll
create you a subdirectory under {\tt AtlasBase/kernel}.  For some
development, though, you will need to modify basefiles that you did not
originate.  For instance, if you wrote a new lapack routine, you would
minimally need to modify a basefile originated by me that creates the lapack
header files.  For these mods, you should ideally send the basefile maintainer
a patch against the basefile, which the basefile maintainer will vet and apply.
If that is impractical, you will need to agree with the maintainer on a
different strategy.  This rule is very important, and I will tend to remove
someone from the group immediately if it is violated.

\item {\bf Always leave the main CVS branch in a working, tested state.}
As completely unfair as it is, I (Clint) am the only guy allowed to break
the main CVS branch.  This means that if you have serious development you'd
like to do under CVS, you need to do a developer branch until everything is
working.  If your mods are all to your own files, you will simply merge the
developer branch back into the main once the new stuff is tested and working.
If they involve other routines, then you'll need to have the person
responsible for the modified codes help with the merge.  So what can you
apply directly to the main trunk?  For instance, if you were maintaining
a kernel, and you improved it, then tested (building the full gemm,
running the testers, etc), and everything is OK, you could
check that in {\bf after} the testing and development process was done.

\item{\bf You only submit on the bug tracker for your own code.}
The ATLAS bug tracker serves roughly the same service for developer
releases the errata does for stable: a place for all confirmed bugs.
You can confirm bugs in your code, but the maintainer of the problematic
code should confirm bugs in code he/she is responsible for.  So, if
you find what you think is a bug in someone else's code, you submit it as a
support request, and the code maintainer should scope it there.  If it is
indeed a bug, the maintainer should move it to the bug tracker, where it
will remain until fixed by a subsequent developer release.  Often, these
"bugs" may turn out to be misunderstandings or bad installs, and thus never
get moved to the bug list.

\item{\bf When in doubt, ask.} If you want to do something that you think
might effect other people, ask them first.  If you are not sure if you should
do something, ask.  If the question is general, post to the developer list.
If it is delicate or personal in some way, you can send to me directly at
{\tt rwhaley@users.sourceforge.net}.
\end{enumerate}

\subsection{Getting CVS write access}
If the developer guidelines seem reasonable to you, and you have code that
is already been accepted into ATLAS, send mail to
{\tt rwhaley@users.sourceforge.net}, saying you want CVS write access, and
agree to the developer guidelines.  Just to make sure you've read them,
I'd like the e-mail to specifically say:
\begin{verbatim}
   I will not do CVS check-ins on files that I am not the maintainer
   for, and I'll keep the main branch in working order.
\end{verbatim}

\subsubsection{I don't agree to the guidelines, what now?}
Send it to the list, or to me if it is delicate.  We ought to be able
to find a way to make things work.

%\subsubsection{How about being a developer and not using CVS for writes?}
%Say you want to write a kernel for the AltiVec, but you don't presently
%have access to a G4.  SourceForge has a compile farm that contains a G4,
%but you don't have compile farm access until you are a member of a SourceForge
%group.  In this case, I'm happy to add you as a developer with your agreement
%that you won't do CVS writes until we agree otherwise.

\subsection{Setting up for CVS write access}
After we agree that you will be an ATLAS SourceForge developer, you need
to go to {\tt www.sourceforge.net}, and sign up as a SourceForge user.
You should see a sidebar with:
\begin{verbatim}
Status:
   Not logged in

   Login via SSL
   New User via SSL
\end{verbatim}

You want to take the {\tt New user via SSL} link.
This gets you a SourceForge user name, which you need to send to me at
{\tt rwhaley@users.sourceforge.net}.  I then have to add you as an ATLAS
developer.

You need to change your CVS access from anonymous/read to developer/write.
It helps if you set your {\tt CVS\_RSH} and {\tt CVSROOT} environment variables
appropriately.  {\tt CVS\_RSH} should be {\tt ssh}.  My {\tt CVSROOT} is
set to 
\begin{verbatim}
   :ext:rwhaley@math-atlas.cvs.sourceforge.net:/cvsroot/math-atlas
\end{verbatim}
Replacing my name (look for {\tt rwhaley} in the above) with yours should
get you what you need.

CVS check in/out will still not work correctly at this date.  Now, do:

\begin{verbatim}
   ssh math-atlas.cvs.sourceforge.net
\end{verbatim}
and enter your user name and
password.  As soon as you get a prompt, logout.  This process creates some 
needed files.

Finally, if you have an ATLAS tree created by anonymous CVS access, the
easiest thing is to delete it and recheck out as yourself.  CVS creates
some files on the first checkout saying what kind of access you are using,
and these will still show as anonymous, despite your {\tt CVSROOT} (this is
true for branching as well, so watch out).  If you
want to scope the CVS commands, I'm sure you can switch over without deleting,
but removing and rechecking out is how I've seen this problem successfully
fixed.

\subsection{Further info}
As I said before, I'm not much of a CVS guru.  I have found 
\begin{verbatim}
   http://ximbiot.com/cvs/manual/stable
\end{verbatim}
to be a helpful in doing CVS stuff.

For SourceForge, everything I know about it came from:
\begin{verbatim}
   https://sourceforge.net/docman/?group_id=1
\end{verbatim}

The companion piece of this guide is the atlas contributer guide, found
in {\tt ATLAS/doc/atlas\_contrib.pdf} of the tarfile.  You really need to 
know that before doing much with this guide.

Extract is explained at:
\begin{verbatim}
   http://www.cs.utsa.edu/~whaley/extract/Extract.html
\end{verbatim}

\section{Adding a LAPACK routine to ATLAS}

It is probably not practical that ATLAS will provide a complete LAPACK
API (as it does with the BLAS) in the foreseeable future, both due to the
algorithmic complexity of some of the operations, and to the sheer number
of routines in LAPACK.  It must be understood that adding routines adds
to the inertia and maintenance costs of the package, and this additional
burden must be offset by real advantage for the user.

ATLAS has so far only added LAPACK routines to ATLAS when we can make a
performance-enhancing algorithm modification.  For instance, we added the
LU and Cholesky factorizations because we used the recursive formulations
of these routines, which provides for better performance on pretty much
any cache-based architecture.

However, when we have added such routines, we usually add the correlated
routines even when a performance advantage is not supplied.  For instance,
upon adding GETRF support, we also added GETRS and GESV.  As far as column-
major routines go, we supply no better algorithm for GETRS or GESV than
LAPACK.  However, since these routines are very simple, and GETRF is very
often used with them, we added them along with GETRF.  The idea here is that
their maintenance costs are not heavy, and real advantage is given to the
user in that we have sped up GETRF, and if the factor and solve are all
he needs, ATLAS will supply a complete solution.

The column-major comment points out another reason to add a routine to
ATLAS: ATLAS supplies the only performance-aware row-major LAPACK
implementation that I am aware of (I'm sure there are some, I just don't
know of any that aren't simply using the col-major stuff, and thus
performing terribly).  It is possible that someone would want to add an
LAPACK routine to ATLAS simply because they need a row-major version, and
someone being motivated enough to write it would probably be ample
justification to add the routine to the ATLAS tarfile.

\subsection{Row-major LAPACK routines}

So far, we have accepted no routines that do not also include a row-major
equivalent, both for BLAS and for LAPACK.  We hope to continue this.  There
are as yet only a few users of the row-major LAPACK/BLAS that I am aware of,
but I believe that this is a chicken/egg problem.

Some people insist on using row-major arrays in C, but if they have access
to a BLAS/LAPACK that supports it, they find the performance is no better
than what they get with simple loops, or that it is calling the col-major
in a naive way, and cutting the problems size they can solve in half by
copying.  Therefore, people with row-major bias don't call the stuff 'cause
it doesn't help them, and the problem continues.

It is my belief, therefore, that good-quality row-major stuff must be
produced before significant demand will appear.  If I'm wrong, I guess
we'll someday drop support for row-major, but I don't think this will be
the case over a long enough time line.

Therefore, despite it being a hassle, having a good quality row-major
implementation is critical for getting an LAPACK routine into ATLAS.
For many routines, since we have row-major BLAS, the algorithm stays
the same, and only some pointer arithmetic need be changed.

Other routines in LAPACK (GETRF is one) have a built in algorithmic bias
towards column-major (in GETRF, this is doing row-pivoting), and another
algorithm with the same stability and usage characteristics should be
employed for row-major (eg., column-pivoting, for GETRF).

\subsection{Outline of Steps}
Here are the general steps to use when adding an LAPACK routine to ATLAS:
\begin{enumerate}
\item Create and debug tester using LAPACK
      \begin{itemize}
      \item Update extractors
      \item Update Makefiles
      \end{itemize}
\item Write and test ATLAS internal routines using above tester
      \begin{itemize}
      \item Update extractors
      \item Update Makefiles
      \end{itemize}
\item Update atlas\_clapack.h to include your new routines
\item Create C and F77 interfaces to your routine
      \begin{itemize}
      \item Update extractors
      \item Update Makefiles
      \end{itemize}
\item Update {\tt clapack.h}
\item Update the LAPACK quick reference guides.
\end{enumerate}

\subsection{Create and debug tester}

The first step in adding a new routine to ATLAS is to create a tester
(and timer) which can be used to verify the correctness of your code.
More than half of the challenge is getting the tester right; with a good
tester/timer, the code usually comes fairly easily.

Your tester will go in {\tt ATLAS/bin} when extracted; you can examine
some of the testers available there to get an idea of what you should
do (eg., look at {\tt ATLAS/bin/[lu/llt/slv/trtri/uum]tst.c}).  All of
these routines come from the basefile {\tt AtlasBase/Clint/atlas-tlp.base},
which is what you should submit your patch against, unless you want to create
your own, separate basefile.

After your tester is written, its column-major components can be tested
against LAPACK by using the {\tt make <rout>tstF} target in
{\tt ATLAS/bin/<arch>}.  You can even test the row-major components by
having the F77 interface transpose the matrices on input, and back on 
output.  See {\tt ATLAS/bin/uumtst.c} for an example of this for square
matrices.

As part of your debugging of the tester, be sure that it not only agrees
that LAPACK produces the right answer, but truly detects errors as well.
For instance, manually overwrite an entry, both in the matrix and in the
padding (in separate tests), and make sure it is caught by the tester.

\subsubsection{Writing {\tt ATLAS/src/testing} C-to-f77 wrapper}
You first need a way for your tester, written in C, to call the LAPACK
routine, written in F77.  All such language translation routines are
kept in {\tt ATLAS/src/testing}, and come from the basefile
{\tt ATLAS/Clint/atlas-ilp.base}.  This wrapper is trivial, though
some of the integer/string stuff is not obvious.  Steal the code from
the other examples.

\subsubsection{Getting your routines extracted}
Now you need to get your files to appear in the right subdirectories,
so you need an entry in the appropriate Make.ext.
All the Make.ext files comes from {\tt AtlasBase/make.base},
so find the rout for your
directory in this file (for examples the line saying \\
{\tt @ROUT ATLAS/interfaces/lapack/F77/src/}) and add your routine name
to the line containing the name of all the other routines.

So, now you are in your working directory (say {\tt ATLAS/src/lapack}),
and you type {\tt make -f Make.ext}, and nothing happens, no new files show up. 
This is because
you need to re-extract your Make.ext file. This can of course be done by
removing your whole ATLAS tree and reinstalling, but less brutally you
can ``just'' use something like this: {\tt extract -b
/home/soender/AtlasBase/make.base -o Make.ext rout=ATLAS/src/lapack
-langM}. The {\tt -langM} switch is required for extract to properly
handle makefiles, so you cannot skip it.

This is the basic procedure for this sort of stuff. When you need a
makefile in a BLDdir subdirectory, the appropriate makefile is copied by 
Make.top from the {\tt ATLAS/makes/} directory. Check Make.ext to see which
basefile they come from, and add your routine name among the names of
the other routines.

Remember to update the Makefiles for both {\tt ATLAS/bin} and
{\tt ATLAS/src/testing}, and to get these makefiles into the appropriate
subdirs.  In order to extract new makefiles, and get them put into the
appropriate subdirs, I typically do something like (from the {\tt BLDdir}:
\begin{verbatim}
   pushd ~/TEST/ATLAS/makes/ ; make -f Make.ext ; cd .. ; \
          make refresh ; popd
\end{verbatim}
(replace the path and arch appropriately, obviously).

\subsection{Create and debug ATLAS internal routines}
The internal LAPACK routines are kept in {\tt AtlasBase/Clint/atlas-lp.base}.
Add your routine here, and update {\tt ATLAS/src/lapack}'s {\tt Make.ext}
and {\tt Makefile} appropriately to build your routine.

You will add your routine in {\tt atlas-lp.base} with an additional
{\tt @ROUT} keyline, but also do not forget to update the include
file {\tt atlas\_lapack.h} at the bottom of the file as well.  You will
need to add your routine to the prototype part, as well as to the 
macro renaming part.  Examine the basefile for details.

Once it is extracting, use your LAPACK-debugged tester to debug your code.

\subsection{Add C and F77 interface routines}
We do this step last, because we don't want to add API routines until
the code is working.  Having debugged and made sure the code is faster
than LAPACK, we're now ready to make it available to the user via the
advertised APIs.  The extracted API files are kept in subdirectories under 
{\tt ATLAS/interfaces/lapack}.

The F77 interfaces are kept in {\tt AtlasBase/Clint/atlas-fint.base}.
Look at the existing examples and
notice how extract generates all four precision from the same routine,
if you use the extract macros. All the code for this interface can be
ripped from LAPACK and adapted. Note that you will usually need to
examine both complex and real versions of the original LAPACK routine,
to find any differences in interface/testing and comments.  You will
also need to remove unneeded EXTERNAL declaration, etc.

This interface does the
parameter checking, and converts any FORTRAN string arguments to some
predefined integer values, and then call the {\tt ATLf77wrap}
interface.  Scope any of the existing routines for details on this.

The C interfaces are easy to write, since they should just check the
input arguments, and then call the ATLAS routine. The codes are stored
in {\tt atlas-clp.base}. Check it out for lots of examples.

\subsection{Update the LAPACK quick reference guides}
The ATLAS user API is defined in the quick reference guides under
{AtlasBase/TexDoc}.  Right now, the supported LAPACK API is small enough to
fit both C and F77 interfaces on one card (single 2-sided landscape page),
but eventually it will be split in two, as with the BLAS quick reference
cards.  Either use the Makefile to do it, or remember to manually throw
the {\tt -tlandscape} flag to {\tt dvips}, and the {\tt -paper a4r} flag
to {\tt xdvi}.

\section{Architectural defaults}
ATLAS's architectural defaults are simply a record of the results of 
a previously run ATLAS search.  They exist for a couple of reasons:
\begin{enumerate}
\item Using architectural defaults, install times are reduced to almost
   bearable levels
\item Because the search is empirical, installs can go wrong if unmonitored.
   Architectural defaults given out in the standard tarfile have at least
   passed the laugh test
\end{enumerate}

\subsection{Rambling on about architectural defaults}
One FAQ for architectural defaults is why any timings are necessary when
using them.  The standard architectural defaults only rarely describe
everything discovered by a search, but rather give only those data
that we feel sure will not vary a great deal.  For instance, for many
machines, the kernels to use, etc., are fully specified, but {\tt CacheEdge}
is not.  CacheEdge varies depending on your L2 cache size, which varies
depending on architecture revision, so it is not specified, allowing it to
tune itself for this variable parameter, while still skipping the search over
less variable things (eg., if the L1 cache or FPU units change, this is
usually a new architecture, not a revision of an old).

That's the theoretical reason why they shouldn't cover all discovered items.
However, ATLAS presently times the kernels in order to be able to
produce a comprehensive SUMMARY.LOG, and these timings {\em could} be skipped,
assuming this functionality were added to the atlas install process.

There are some weaknesses of architectural defaults.  One of the main ones
is how they can go out of date, and cause slowdown.  One big way this can
happen is with compiler changes.  For instance, gcc 3.0 produces completely
different (and inferior) x86 code than the 2.x series, and 4.0 was similarly
worse than latter-day gcc 3.  Almost all architectural defaults in ATLAS 3.8
are compiled with gcc 4.2.  

Anytime a different compiler is used, the architectural defaults become
suspect.  For truly inferior compiler (like gcc 3.0 or 4.0), there is no
way to get good performance, but at least some problems can be worked
around by having ATLAS adapt itself to the new compiler, and architectural
defaults prevent this from happening.

\subsection{Making your own architectural defaults}
This section describes how to create architectural defaults as of ATLAS
3.7.12 and later.  For older releases, the process is similar, but not quite
the same, and is covered in the older {\tt atlas\_devel} available in
those tarfiles.
\begin{enumerate}
\item Get an install, correct in all details, that you want to immortalize.
\item cd to your {\tt OBJdir/ARCHS} directory
\item Type {\tt make ArchNew}
\end{enumerate}

This will copy the search result output files into a directory
{\tt <OBJdir>/ARCHS/<MACH>/}, with appropriate subdirs under
that.  You can then go into these guys and delete files you don't want to
be part of the defaults (eg., {\tt atlas\_cacheedge.h}, etc).

Now, to save these defaults to a transportable format, you can have the
makefile create the tarfile for you by:
\begin{verbatim}
   make tarfile
\end{verbatim}

\subsection{Getting ATLAS to use your shiny new defaults}
Pretty easy:  
\begin{enumerate}
\item Run configure, creating {\tt Make.inc}, but do not start the install.
\item Take the tarfile you created, and copy it under
   {\tt ATLAS/CONFIG/ARCHS} source directory.  
\item Edit your {\tt Make.inc} and make sure the {\tt ARCH} macro matches
      the base name of your tarfile (eg., P4ESSE3), and that the INSTFLAGS
      macro has the flags {\tt -a 1} (do use arch defs).
\item Continue the install as normal (eg. {\tt make build}).
\end{enumerate}

\section{Sanity testing for an ATLAS install}
From ATLAS3.3.8 forward, ATLAS has had a ``sanity test'', which just does
some quick testing in order to ensure that there are no obvious problems
with the installed ATLAS libraries.  It runs all of the standard BLAS interface
testers, with the default input files, and it then runs a few fixed cases
of ATLAS's lapack tester routines (eg., {\tt ATLAS/bin/invtst.c}, etc).  The
advantage of these lapack testers is that they depend on many of
the BLAS as well as the lapack routines, so you get a lot of testing for
a minor amount of time.  The sanity checks do not require any non-ATLAS
libraries for testing, so the only dependence that a user who has installed
ATLAS may not be able to satisfy is the need for a Fortran77
compiler, which is required for the BLAS interface testers.
As of ATLAS3.7.12, ATLAS can also run a reduced set of tests for users
who do not have a fortran compiler.

\subsection{Invoking the sanity tests}
These tests are invoked from your install directory by:
\begin{verbatim}
    make check
\end{verbatim}

If you are using threads, you will want to run the same tests for threading
via:
\begin{verbatim}
    make ptcheck
\end{verbatim}

\subsection{Understanding the sanity test output}
Once you fire off this tester, you'll see a lot of compilation going on.
All compilation is done up front, and then the testers are run at the end.
All tester output is dumped to some files (we'll see specifics in a bit),
which are then automatically grepped for errors at the end of the run.  It
is the results of this grep that the user will see.  For example, here's
the output from a run on my Athlon running Linux:
\begin{verbatim}
dudley.home.net. make check
...
... bunch of compilation ...
...
DONE BUILDING TESTERS, RUNNING:
SCOPING FOR FAILURES IN BIN TESTS:
fgrep -e fault -e FAULT -e error -e ERROR -e fail -e FAIL \
        bin/Linux_ATHLON/sanity.out
8 cases: 8 passed, 0 skipped, 0 failed
4 cases: 4 passed, 0 skipped, 0 failed
8 cases: 8 passed, 0 skipped, 0 failed
4 cases: 4 passed, 0 skipped, 0 failed
8 cases: 8 passed, 0 skipped, 0 failed
4 cases: 4 passed, 0 skipped, 0 failed
8 cases: 8 passed, 0 skipped, 0 failed
4 cases: 4 passed, 0 skipped, 0 failed
DONE
SCOPING FOR FAILURES IN CBLAS TESTS:
fgrep -e fault -e FAULT -e error -e ERROR -e fail -e FAIL \
        interfaces/blas/C/testing/Linux_ATHLON/sanity.out | \
                fgrep -v PASSED
make[1]: [sanity_test] Error 1 (ignored)
DONE
SCOPING FOR FAILURES IN F77BLAS TESTS:
fgrep -e fault -e FAULT -e error -e ERROR -e fail -e FAIL \
        interfaces/blas/F77/testing/Linux_ATHLON/sanity.out | \
                fgrep -v PASSED
make[1]: [sanity_test] Error 1 (ignored)
DONE
\end{verbatim}

So, in the LAPACK testers we see no failures (all tests show 
{\tt 0 failed}), and we have no output from the BLAS testers, which is
what we want.  Notice the lines like:
\begin{verbatim}
   make[1]: [sanity_test] Error 1 (ignored)
\end{verbatim}

This is due to fgrep's behavior, and does not indicate an error.  If fgrep
does not find any pattern matches, it returns a 1, 0 on match.  Therefore,
since we are grepping for error, getting an ``error condition'' of 1 is
what we hope for.

\subsection{Finding the context of the error}
If the sanity test ouput shows errors, the next step is to track down where
they are coming from.  You can see in the output the files that are being
searched for errors.  They are:
\begin{verbatim}
        bin/sanity.out
        interfaces/blas/C/testing/sanity.out 
        interfaces/blas/F77/testing/sanity.out 
\end{verbatim}

The threaded sanity test uses the same filenames with {\tt pt} prefixed.

The first thing to notice is which of these tests are showing errors.
The testers in bin are higher level than those in the interfaces directories,
so if you get errors in both, track down and fix the interface errors first,
as they may be causing the lapack errors.  If both C and F77 BLAS interfaces
are showing errors, I always scope and fix the Fortran77 stuff first, since
Fortran is simpler (no RowMajor case to handle).  Only if an error only
shows up in C testing do I scope that output instead of the Fortran77.

The grepped error message probably gives you no idea what actually went wrong
(it may show something as simple as:
\begin{verbatim}
    FAIL
\end{verbatim}
for instance), so you must go look at the {\tt sanity.out} in question.
For instance, you might need to scope
{\tt interfaces/blas/F77/testing/sanity.out}.  You do a search for
whatever alerted you to the problem (eg., {\tt FAIL}), and you see by the
surrounding context what tester failed.

\subsection{Tracking down an error in the BLAS interface testers}
The BLAS testers are split by BLAS Level (1, 2 or 3) and precision/type
(s,d,c,z).  The basic names of the tester executables are 
\begin{verbatim}
    x<pre>blat<lvl>
    x<pre>cblat<lvl>
\end{verbatim}
for Fortran77 and C, respectively.  The Level 1 testers 
({\tt x[s,d,c,z]blat1}) test certain fixed cases, and thus take no input file.
So if the error is in them, you simply run the executable with no args in
order to reproduce the failure.

The Level 2 and 3 testers allow a user to specify what tests should be run,
via an input file.  The standard input files that ATLAS runs with are:
\begin{verbatim}
   <pre>blat<lvl>.dat
   c_<pre>blat<lvl>.dat
\end{verbatim}
respectively.  The format of these input files is pretty self explanatory,
and more explanation can be found at:
\begin{verbatim}
   www.netlib.org/blas/faq.html
\end{verbatim}
To run the tester with these files, you redirect them into the tester.  For
instance, to run the double precision Level 2 tester with the default input
file, you'd issue:
\begin{verbatim}
   ./xdblat2 < ~/ATLAS/interfaces/blas/F77/testing/dblat2.dat
\end{verbatim}

You should be aware that only the first error report in a run is accurate:
one error can cause a cascade of spurious error reports, all of which may go
away by fixing the first reported problem.  So, it is important to find and
fix the errors in sequence.

I usually copy the input file in question to a new file that
I can hack on (for instance, if the error was in the double precision Level 2,
I might issue:
\begin{verbatim}
   cp ~/ATLAS/interfaces/blas/F77/testing/dblat2.dat bad.dat
\end{verbatim}
I then repeatedly run the
routine and simplify the input file until I have found the smallest, simplest
input that displays the error.

The next step is to rule out tester error.  The way I usually
do this is to demonstrate that the error goes away by linking to the Fortran77
reference BLAS rather than ATLAS (you can only do this for errors in the
F77 interface, obviously).  I usually just do it by hand, i.e., for the
same example again, I'd do:
\begin{verbatim}
   f77 -o xtst dblat2.o /home/rwhaley/lib/libfblas.a
\end{verbatim}
If the ATLAS-linked code has the error, and this one does not, it is a
strong indication that the error is in ATLAS.  If the F77 BLAS are shown
to be in error, it is usually a compiler error, and can be fixed by turning
down (or off) the optimization used to compile the tester.

Now you should have confirmed the tester is working properly, and that
the error is in a specific routine (let us say DNRM2 as an example).
As a quick proof that DNRM2 is indeed the problem, you can link explicitly
to the F77 version of DNRM2, and to ATLAS for everything else (see
Section~\ref{sec-GoodBlas} for hints on how to do this).  If this
still shows the error, you are confident that ATLAS's DNRM2 is indeed causing 
the problem, and you should either track it down, or report it (depending on
your level of expertise).

\subsection{Tracking down an error in the {\tt bin/} testers}
\label{sec-LapackDebug}

The sanity tests only run the LAPACK testers in this directory.  The LAPACK
routines depend on the BLAS, so ignore errors in lapack testers until all
the BLAS pass with no error.  If you have errors in LAPACK but the BLAS pass
all tests, then you have to hunt for the error in the LAPACK routines.

First, rule out that it's not a problem in the BLAS that is just not showing
up in the BLAS testing.  Get yourself a reference BLAS library, as explained
in Section~\ref{sec-GoodBlas}.  Then, set your {\tt Make.inc}'s {\tt BLASlib}
macro to point to the created reference BLAS library.  Then, you need to
compile a library that uses ATLAS's lapack routines, but the reference
BLAS.  This can be done by compiling the same executable name with {\tt \_sys}
suffixed.  For instance, if you were running the LU tester, {\tt xdlutst},
you would say {\tt make xdlutst\_sys}, and then run this executable with
the same input.

If the error goes away, then the error is really in the ATLAS BLAS somewhere.
I then usually look at the LAPACK routine and tester in question to find out
what its BLAS dependencies are, and manually link in the reference BLAS
object files until I find the exact BLAS causing the problem.  Usually
once you know what routine causes the prob, you can reproduce the error
with the BLAS tester (i.e. you need a IDAMAX call with N=12, incX=82).

If the error still persists using ATLAS's LAPACK and the Fortran77 BLAS,
the next trick is to do LAPACK just like the BLAS: download and compile the F77
LAPACK from netlib ({\tt www.netlib.org/lapack/lapack.tgz}).  You then
set your {\tt Make.inc's} {\tt FLAPACKlib} to point to your Fortran77 lapack
library.  You then suffix the base executable name with {\tt F\_sys} (eg., for
LU again, you would do {\tt make xdlutstF\_sys}), and
you will get a tester linked against the Fortran77 BLAS {\em and} LAPACK.
If this also shows to be in error, there is an error in the tester, or in
the compiler.  Try turning down compiler optimization to rule in or out
compiler errors.

\section{Antoine's testing scripts}

Before a stable release, we always do as much testing as possible.  The
900 pound gorilla of testers is Antoine's tester scripts.
This tester can run as long as several days, and does a great number of
both fixed and random tests, and if it completes with no errors, you have
a pretty good
idea that the code is fairly solid. 
Even the casual
user ought to run the sanity testing as a matter of course, and that should
always be ran and passed first.  Also, much of the methodology for 
understanding output, tracking down problems, etc, is the same for this
tester and the sanity test, so read those sections first for tips I will
not bother to repeat here.

\subsection{Setting up and installing the tester}
First, you need to get the tester tarfile.  You can get it from the file 
releases on sourceforge, or, if you are using CVS, you can checkout the
{\tt AtlasTest} module.  You then untar this guy in the directory you
want it ({\tt bunzip2 -c atlas\_test1.1.3.tar.bz2 | tar xvmf -}).

Now, you create a directory for each architecture you wish to run
the tester on, using the configure command.  For instance, I could
create a subdirectory under my AtlasTest directory with the following
commands (following the above untar):
\begin{verbatim}
   cd AtlasTest
   mkdir Core2DuoSSE3
   ../configure --atldir=/home/whaley/TEST/ATLAS3.7.36.0/obj64/
\end{verbatim}
Where of course {\tt --atldir} provides the path to the {\tt BLDdir} that
you want to test.  From here on out, we will call this directory, which
you have configured for a particularl platform's test, as the {\tt TSTdir}.

Some of these tests need a reference BLAS library to compare against, so
you need to fill in your ATLAS install's {\tt BLDdir/Make.inc} with a
trusted, complete {\tt BLASlib}.  See the following section for details on
this.

You are now ready to start the testing, as described in the following sections.

\subsection{Getting a good {\tt BLASlib}}
\label{sec-GoodBlas}

Some of these tests need a reference BLAS library to compare against, so
you need to fill in your ATLAS install's {\tt BLDdir/Make.inc} with a
trusted, complete {\tt BLASlib}.  On modern machines, we typically just
compare against the Fortran77 reference BLAS from netlib, though this
makes the install run longer.  On slower machines, you may need to use
an optimized/vendor BLAS to do testing, but then when you find errors
you will need to debug whether it is ATLAS or the optimized BLAS that
are causing the problem.

Get the BLAS reference tarfile from {\tt www.netlib.org/blas/blas.tgz}.
then do something similar to the following:
\begin{verbatim}
   mkdir FBLAS
   cd FBLAS
   gunzip -c ../blas.tgz | tar xvf -
   gfortran -O -c *.f
   ar r ~/lib/libfblas.a *.o
\end{verbatim}

You may need to substitute for your Fortran77 compiler and flags, and if your
system uses ranlib, run that on {\tt libfblas.a} as well.  It is important
the Fortran77 compiler and flags used to compile this library match those
used by ATLAS!

Now simply set your {\tt Make.inc's} BLASlib to something like:
\vspace*{-0.1in}
\begin{verbatim}
   BLASlib = /home/rwhaley/lib/libfblas.a
\end{verbatim}

\subsubsection{Using an optimized BLAS}

You may want an optimized library if one
is available, since the Level 3 tests can go on for much longer if
you use only the reference library.  However, only a few vendor libraries
supply all of the BLAS that ATLAS provides (to be fair, ATLAS provides
BLAS above those mandated by the standard; it provides all the routines
present in the Fortran77 reference library).  So, the easiest way to
get a complete library is to also install the reference Fortran77 library
from netlib, as described in the previous section.

Now, you can  set {\tt BLASlib} so that the optimized library is linked in
first, and the reference BLAS are used for any routines not provided in
the optimized library.  For instance, here's an old {\tt BLASlib} for using
MKL:
\begin{verbatim}
   BLASlib = /home/rwhaley/lib/libmkl32_def.a /home/rwhaley/lib/libfblas.a
\end{verbatim}

For many routines, the tester cannot tell the difference between an error
in the BLAS given by {\tt BLASlib}, and an error in ATLAS.  Subsequent
section will explain how to figure this out, but understand that
a lot of optimized BLAS will fail this tester, in which case you need
to link against the F77 BLAS instead of the optimized version of that
routine.  Let us say you find out that there are errors in the
optimized {\tt DTRSM}.  In this case, you can simply link in the F77 reference
DTRSM object file first to override the on in the optimized lib.  So, your
BLASlib line would then look something like:
\begin{verbatim}
   BLASlib = /home/rwhaley/FBLAS/dtrsm.o \
             /home/rwhaley/lib/libmkl32_def.a /home/rwhaley/lib/libfblas.a
\end{verbatim}

Obviously, if you have more than a few routines like this, just testing
against the f77 reference BLAS and taking the extra runtime is the way to
go.


\subsection{Running the tester}
The first thing to be aware of in running the tester is that the log files
it creates can take up a {\em lot} of space.  You can kill the log files
as soon as the tester finishes, but you need enough space for it to complete.
The command to run the tester is simple:
\begin{verbatim}
   make
\end{verbatim}

As previously mentioned, however, this tester can run as long as several
days.  So, if you are connected to the machine with an unreliable or short-
term connection, you will need to ensure it can continue to run even if
you are disconnected.  Under most unixes, you can do this by using the
{\tt nohup} command.  For example:
\begin{verbatim}
   nohup make |& tee PPRO.out &
\end{verbatim}
is what I use with the tcsh shell.  Bourne shell uses users will need a
different redirect command.

\subsection{Finding errors}
Some errors you may see on standard out, or in the log file.  If you haven't
seen any there, you need to scope the stored up output created by the tester.
The tester puts such output files in {\tt TSTdir/res}.  There's a
small shell script {\tt AtlasTest/scope.sh} which, when run from
{\tt TSTdir} will grep the relavent files for errors.  If it finds an error,
you then edit the file in question ({\tt scope.sh} prints the file it is
grepping),
and find the test run that caused the error.  This can take a bit what with
the volume of output, but is doable if you stick with it.

Once you have the error, you need to repeat it.  You can try running the
exact case, but sometimes that won't do it (for instance, you have a memory
error that requires you to run many cases); you then need to find a small
run that does demonstrate the error.

You should then apply the normal tricks (linking to F77 BLAS instead of
sys blas, having the tester call the f77 blas twice, etc) to ensure the
error really is in ATLAS, before tracking the error to its source.

\subsection{Tracking down errors in the {\tt bin/} testers}
There are two types of bin/ testers: lapack and blas.  The BLAS testers
have executable names of the form 
\begin{verbatim}
   x<pre>l<level>blastst
\end{verbatim}
The BLAS testers test ATLAS against a known-good implementation, so the first
thing to do is make sure the error is in ATLAS, and not the known-good
implementation.  To do this, compile the reference BLAS from netlib (using
conservative compiler flags), as 
discussed in Section~\ref{sec-GoodBlas}, and then relink and rerun the
test in question.  If the error goes away, you have found an error in your
known-good library, not ATLAS.  If it stays, you have found an error in
ATLAS, and you should track it down or report it.
See Section~\ref{sec-LapackDebug} for information on tracking problems in the
LAPACK testers.

\section{Finding a good NB for GEMM}
One of the things I do most frequently with user-submitted kernels is reduce
the blocking factor that the user has chosen.  I often choose smaller NB
than the best for asymptotic GEMM performance, and even more often choose
one that does not yield the best performance in the kernel timer.
To understand why, you must understand the following points, explained in
turn below:
\begin{enumerate}
\item Better kernel timing (eg. {\tt make ummcase} in your 
{\tt <OBJdir>/tune/blas/gemm/} directory) does not always yield better
total GEMM performance
\item Large NB means significantly more time in cleanup code
\item Large NB means significantly more time in unblocked application code
\end{enumerate}
\subsection{Better kernel timing does not always yield faster GEMM}
The kernel timer (invoked by one of the {\tt make mmcase} variants
available in {\tt <OBJdir>/tune/blas/gemm/}) tries to mimic the way
ATLAS calls the kernel.  However, it does not do everything the same way.
First, there is no cleanup, so it is always calling the kernel only.  More
importantly, {\tt CacheEdge} has not yet been determined, so no
Level 2 Cache blocking is being performed.  Therefore, it may sometimes
look like you are better off to block the kernel for the L2 when using
these kernel timers, when in fact, if you instead block for the Level 1 cache,
CacheEdge will then further speed things up later, and thus the smaller
NB achieves better GEMM performance, even when it runs slower in the kernel
timer.  

For machines with very large L1 caches, often several blocking factors that
fit into L1 have roughly the same performance.  In such a case, it is very
likely that you want to choose the smallest achieving that rough
performance, as it will allow more blocks to fit into the L2 blocking
to be done later.

If a kernel appears to get much better performance with a large NB, the
best idea is to build a full GEMM using both the best-performing small
NB, and the best performing large NB, and seeing what the gap truly is.
Very often, the small kernel will actually be better even asymptotically,
and if it is not, it will often be so much better for smaller problems
that it makes sense to use it anyway.

Even beyond these explanations, it is sometimes the case that the kernel
timer predicts good performance that is not realized when the full GEMM
is built.  This is usually due to inadequate cache flushing, leading
to overprediction of performance because things are retained more in
the cache than they are in practice.  Therefore, I usually pump up the
flushing mechanism (set {\tt L2SIZE} of your {\tt Make.inc} to ridiculously
large levels).  No matter what, actual full GEMM performance is the final
arbiter.  If it is not as high as predicted by the kernel timer, it
may be worthwhile to see if other, smaller NB, cases achieve the same
full-gemm performance.

\subsection{Large NB means more time in cleanup}
One bad news about choosing a large NB is that applications will spend
more of their time in cleanup.  Let us say you choose a block factor of
120.  In this case, many applications will never even call your optimized
kernel, but spend all their time in GEMM cleanup.  Some applications are
staticly blocked, and if their NB is smaller than yours, they can spend
their entire time in cleanup even for large problems.

Therefore, if you must choose a large NB in order to get adequate GEMM
performance, you must pay an unusual amount of attention to cleanup
optimization.  However, as the next section will discuss, even if
cleanup ran at the same speed as your best kernel, this will yield
poor performance for many codes.

\subsection{Large NB means more time in unblocked application code}
Probably the worst thing about choosing a large NB is that many applications
use Level 1 and 2 BLAS in order to do the unblocked part of the
computation.  These BLAS are usually at least an order of magnitude slower
than GEMM.  Therefore, as you increase NB, for applications with unblocked
portions, you increase the proportion of time spent in this order-of-magnitude
slower code.  Therefore, even with perfect cleanup, a large NB may result
in an application running {\bf at less than half speed},
even though GEMM performance is quite good.  

To get an idea of this, simply scope the factorizations provided by LAPACK.
These applications are staticly blocked, so that 
the column factorizations (eg., DGETF2 for LU) are used until NB is reached.
If ILAENV returns a blocking factor smaller than your GEMM, the applications
will stay in cleanup even for large problems.  Even worse, some applications
(eg., QR) require workspace proportional to NB, and since dynamic memory
is not used, it is possible even if you hack ILAENV to use the correct
blocking factor, they will be forced to a smaller one.

\subsection{Finding a good NB}
I will call the first level of cache accessed by the floating point unit
the Level 1 cache, regardless of whether it is the first level of cache
of the machine (there are a number of machines, such as the P4 (prescott)
and Itanium where the FPU skips the Level 1 cache).  Let $N_e$ be the
number of elements of the data type of interest in this cache.  If this
cache is write-through, then a rough guess for a good upper bound is
$N_B \le \sqrt{N_e}$.  If the cache is not write-through, this is still
the upper bound, but many larger caches often benefit from using a smaller
$N_B$, one roughly $N_B < \frac{\sqrt{N_e}}{3}$.  We can describe this
more exactly, but these bounds are easy to compute during tuning.  

You should not choose an $N_B$ that is a power of 2, as this could occasionally
cause nasty cache conflicts.  There's often a small advantage to choosing
$N_B$ that are a multiple of cache line size; this can sometimes be critical,
depending on the arch.

So, the basic idea is to start looking at $N_B$ given by the above two
computations, and then try a little smaller and larger using the kernel
timer.  If you get two that tie for out-of-cache performance, always take
the smaller.  If best performance is achieved with very large $N_B$ (say
$N_B \ge 80$), then always confirm that it yields better GEMM performance
than a smaller $N_B$, and that application performance is not severely
impacted, particularly for smaller problems.

The way I usually time application performance is to time ATLAS's LU.
This actually gives you a very rosy picture of how a large block factor
will effect performance, in that it uses recursion rather than staticly
blocking.  This means that ATLAS's LU does not have any unblocked code,
and thus doesn't slow down the way LAPACK's LU will for large $N_B$.
However, if even this code shows performance loss for smaller sizes,
you know your cleanup needs to get a {\em lot} better, or you need
to reduce $N_B$, even if it results in a slight reduction in GEMM
performance.
If you want to get a better idea of how most applications will perform,
time one of LAPACK's factorizations instead.

Under no circumstances should you choose a blocking factor much larger
than 120.  I confine the ATLAS search to a maximal size of 80 for the
above reasons, but occasionally go a little higher for machines without
effective L1 caches.  However, this can absolutely kill application
performance.  Further, it is never a good idea to completely fill
an Level 2 cache with your block.  It may look good in GEMM, but it
will die in any application, both for the reasons above, and the
following: The L2 cache is shared instruction/data.  Filling it
with data will often lead to instruction loading/flushing cycle
when a larger application is calling.  Remember that GEMM is of
interest because of all the applications that are built from it,
not when used in isolation.

If a NB larger than 60 only gives you a few percent, always choose a 
smaller one; only go above 80 for significant advantage, and essentially
don't go above 120 unless absolutely necessary, and then you can expect
slowdown in many applications, even once you have fully optimized all
cleanup cases.

\newpage
\section{Information on atlconf}
{\bf NOTE:} this information was out of date before it was finished, so
this discussion should be viewed as an introduction only.

For ATLAS 3.7.12, ATLAS's configure routine was completely rewritten for
greater modularity.  The total amount of code probably increased, but the
amount that must be examined at any time should be very much smaller. 

In the new system, the topmost unit is {\tt ATLAS/configure} which is a 
BFI shell script which allows ATLAS's {\tt config.c} to be invoked in a
way very similar to gnu configure.  This shell script gathers some info
and fills in a Makfile which is then used to build {\tt xconfig} from
{\tt ATLAS/CONFIG/src/config.c}.  {\tt config.c} is a driver program
that first calls various probes to determine any information not overridden by
user flags, and then calls {\tt xspew} to create a full {\tt Make.inc} for
the target architecture.  {\tt xsprew} is built from the file 
{\tt ATLAS/CONFIG/src/SpewMakeInc.c}.  

The idea is to change ATLAS's install so it consists of the following
commands:
\begin{enumerate}
\item {\tt /path/to/ATLAS/configure} : Create {\tt Make.inc} and build subdirs
      in the present directory (ATLAS no longer requires building in arch-spec
      directories under the source tree)
\item {\tt make build} : Build ATLAS
\item {\tt make check} : run sanity tests
\item {\tt make time} : run simple benchmarks, compare observed vs. expected
      performance, and issue warning if too low
   \begin{itemize}
   \item Keep record of arch default installs with perfermance as \% of
         clock rate (not peak!)
   \item If arch def used wt bad compiler, this will detect performance
         difference, and user should be warned
   \item If no arch defaults (so expected performance unknown), still create
         standard benchmark output file for submission to ATLAS
   \end{itemize}
\item {\tt make install} : copy libraries and include files to user-specified
      directories
   \begin{itemize}
   \item Will need to figure out what-all include files to copy!
   \end{itemize}
\end{enumerate}

\subsection{Weaknesses in {\tt spew}/{\tt config}}
\begin{enumerate}
\item Needs a flag for the {\tt delay} variable
\item Needs correct lib setup:
   \begin{itemize}
   \item Spew \& config flags for BLASlib, FBLASlib, FLAPACKlib, extra syslibs.
   \item config.c needs an updated GetSysLib, output passed to spew 
         via extra syslibs.
   \item Need flags for archiver, archflags, and ranlib 
         (also table lookup in config.c)
   \end{itemize}
\end{enumerate}

\subsection{Probe Overview}
From ATLAS 3.7.12 on, ATLAS's config routine was rewritten for greater
modularity, with each config probe having its own driver and so on.
For this discussion, we will refer to the machine doing the cross-compiliation
as the {\it frontend} (abbreviated as FE), and the machine which ATLAS is
being tuned for the {\it backend} (abbreviated as BE).  Note that if you are
not doing cross-compilation (the majority of the time) the front-end and
back-end are the same machine.

Every type of probe has a frontend driver (occasionally, config may directly
call the backend driver, if there is only one) which will itself call
multiple backend drivers.  
For instance, the probe to compute the architecure runs on the frontend, and
calls different backend drivers depending on the assembly dialect and operating
system of the backend.
The files for the frontend drivers are located in
{\tt ATLAS/CONFIG/src}, and the backend files are in 
{\tt ATLAS/CONFIG/src/backend}, with all include files in 
{\tt ATLAS/CONFIG/include}.  All frontend probes use the file 
{\tt atlconf\_misc.c} (prototyped in {\tt atlconf\_misc.h}), 
which handles things like file I/O, issuing shell commands, etc.  
The current probes used by config are:

\begin{enumerate}
\item {\bf OS Probe}
   \begin{description}
   \item [Purpose:] Discover the Operating System being used
   \item [Inputs:]  None
   \item [Outputs:] Enumerated type of OS
   \item [FE files:] {\tt probe\_OS.c}
   \item [BE files:] None (uname on BE)
   \end{description}
\item {\bf Assembly dialect probe}
   \begin{description}
   \item [Purpose:] Discover what ATLAS assembly dialect works
   \item [Inputs:]  OS enum (gives subdialect of assembler)
   \item [Outputs:] Enum of assembly dialect
   \item [FE files:] {\tt probe\_asm.c}, 
   \item [BE files:] {\tt probe\_this\_asm.c} -- [{\tt probe\_gas\_parisc.S}, 
         {\tt probe\_gas\_ppc.S}, {\tt probe\_gas\_sparc.S}, 
         {\tt probe\_gas\_x8632.S}, {\tt probe\_gas\_x8664.S}]
   \end{description}
\item {\bf Vector ISA extension probe -- assembly}
   \begin{description}
   \item [Purpose:] Discover which of supported vector ISA extensions work
   \item [Inputs:]  enums for OS and assembly dialect
   \item [Outputs:] iflag = \verb+( (1<<ISA0) | (1<<ISA1) | ... | (1<<ISAn) )+
   \item [FE files:] {\tt probe\_vec.c} 
   \item [BE files:] {\tt probe\_svec.c} --
                  [{\tt probe\_AltiVec.S}, {\tt probe\_SSE.S}],
                  {\tt probe\_dvec.c} -- [{\tt probe\_SSE2.S}],
                  {\tt probe\_dSSE3.c} --[{\tt probe\_SSE3.S}]
   \end{description}
\item {\bf Vector ISA extension probe -- C}: Write this later, using C-inline
      statements for platforms where we don't speak the assembly, but can still
      use peter's vector include file
\item {\bf Architecture probe}
   \begin{description}
   \item [Purpose:] Discover target architecure/machine
   \item [Inputs:]  OS and assembly enums [force 64/32 bit usage]
   \item [Outputs:] enum of arch
   \item [FE files:] {\tt archinfo.c}
   \item [BE files:] {\tt archinfo\_x86.c}, {\tt archinfo\_linux.c}, 
         {\tt archinfo\_freebsd.c}, {\tt archinfo\_aix.c},
         {\tt archinfo\_irix.c}, {\tt archinfo\_sunos.c}
   \item [Notes:] See Section~\ref{sec-archProbe} for more details.
   \end{description}
\item {\bf 64-bit probe}
   \begin{description}
   \item [Purpose:] Discover if arch supports 64-bit pointers
   \item [Inputs:]  OS, arch [user choice]
   \item [Outputs:] 32 / 64
   \item [files:] Config directly calls archinfo
   \item [Notes:]: New policy: config assumes whatever compiler gives you w/o
                   -m32 -m64, and user must throw special flag to append these
                   to the line.
   \end{description}
\item {\bf Compiler probe}
   \begin{description}
   \item [Purpose:] Find good compilers
   \item [Inputs:]  OS, arch [,suggested compilers]
   \item [Outputs:] The following:
      \begin{enumerate}
      \item F2CNAME, F2CINT, F2CSTRING enums
      \item Compilers and flags
      \end{enumerate}
   \item [FE files]: {\tt probe\_comp.c} {\tt probe\_f2c.c} {\tt probe\_ccomp.c}
   \item [BE files]: {\tt f2cname[F,C].[f,c]}, {\tt f2cint[F,C].[f,c]},  
                     {\tt f2cstr[F,C].[f,c]}, 
                     {\it ccomp interaction not yet done}
   \item [Note:] This is complex, see Section~\ref{sec-compProbe} for details.
   \end{description}
\item {\bf Arch defaults probe}
   \begin{description}
   \item [Purpose:] Discover arch defaults
   \item [Inputs:]  OS, arch, compilers
   \item [Outputs:] Whether to use arch defs ({\tt INSTFLAGS} in {\tt Make.inc}
   \item [files:] {\tt ARCHS/Makefile}
   \item [invoke:] Arch default setup is instigated by {\tt atlas\_install.c}.
   \item [notes:] May want to have it autobenchmark kernel, test against
         table of expected perf, to see if arch def are OK wt this compiler
         version.
   \end{description}
\end{enumerate}

\subsection{Architectural Probes}
\label{sec-archProbe}
{\samepage
We use the {\tt archinfo\_xxx} probes to discover the following architectural
information:
\begin{itemize}
\item 'n': number of cpus
\item 'c': number of cache levels
\item 'C' \#: size in KB of cache level \#
\item 'v': verbose (prints strings as well as ints)
\item 'm': clock rate in Mhz
\item 'a': ATLAS architecture classification 
\item 'b': support for 64 / 32 bits
\item 't': Is cpu throttling currently on (-1: no, 0: don't know, 1: yes)
\end{itemize}
If a given probe cannot find that particular item, it is returned as 0.
}

The frontend wrapper script {\tt archinfo.c} calls these
guys according to OS, and tries to get all flags filled in with union of
functionality of archinfo\_x86 and archinfo\_$<$OS$>$.

\subsection{Notes on configure}
New policies:
\begin{itemize}
\item Any 64-bit arch defaults to building whatever the compiler natively does
      w/o flags unless overridden
\item L2SIZE always set to 4MB unless overridden
\item Timer defaults to standard wall/cpu unless overridden
   \begin{itemize}
   \item Exception is solaris, where we default to solaris hr timers
   \item If Mhz is passed in, use cycle-accurate wall-timer on x86
   \end{itemize}
\end{itemize}

Deprecated machines (no longer supported in config or arch def):
\begin{itemize}
\item Alpha chips (some EV6 may still be alive -- UTK?)
\item All pre-G4 PowerPC chips
\item All pre-UltraSPARC Sun chips
\end{itemize}

Still missing HPUX support.  Linux and FreeBSD support best tested.

\subsection{Compiler Setup and Handling in ATLAS Config}
\label{sec-compProbe}

This is complicated as hell.  Potentially, each architecture/OS combo has
unique compiler and flags for each supported compiler (more below), and the 
user can override any/all of these.  I'm changing the number of supported
compilers for greater flexability.  These are:
\begin{description}
\item [{\tt ICC}] : compiles all C interface routines.
      Since it is not used for any kernel compilation
      the performance impact of this compiler should be minimal.
\item [{\tt SMC}] : used to compile ATLAS single precision matmul kernels
\item [{\tt DMC}] : used to compile ATLAS double precision matmul kernels
\item [{\tt SKC}] : used to compile all non-interface, non-gemm-kernel
                    single precision ATLAS routines
\item [{\tt DKC}] : used to compile all non-interface, non-gemm-kernel
                    double precision ATLAS routines
\item [{\tt XCC}] : used to compile all front-end codes
\item [{\tt F77}] : Valid fixed-format Fortran77 compiler that compiles
      ATLAS's F77 interface routines.  This should match the Fortran77
      the user is using.  This compiler's performance
      does not affect ATLAS's performance, and so may be anything.
\end{description}

%In the new config, I try to make everything a separate probe, so that
%someone can use their own config, and figure a specific thing out w/o examining
%a huge code base.  Right now, the compiler stuff is built into config, so
%I should change this.  Due to volume of outputs, may need to write to a file
%rather than stdout, and let config read file.


Here's my present design:
\begin{enumerate}
\item {\bf Compiler defaults}: are read in from {\tt atlcomp.txt}, which
      allows the user specify default compiler/flags, as well as specific
      ones for particular architectures, and multiple compilers for a given
      arch.  
   \begin{itemize}
      \item {\tt ATLAS/SRC/probe\_comp.c}
   \end{itemize}
\item {\bf Executable search}: takes name of executable (in this case
      a compiler name), and finds the path to it.  Skipped if the user
      provides the path as part of the compiler.
   \begin{itemize}
   \item {\it unimplemented}: presently use compilers w/o path
   \end{itemize}
\item {\bf C compiler interaction probe}: separate probe that takes two or
      more C compilers and their flags as arguments, and makes sure they
      are able to call each other w/o problems.
   \begin{itemize}
   \item {\it unimplemented}: presently assumed to work
   \end{itemize}
\item {\bf F77/C calling convention probe}: as in present config, but as
      an independent probe.
   \begin{itemize}
   \item {\it done}: front-end is {\tt probe\_f2c.c}
   \end{itemize}
\end{enumerate}

\end{document}
